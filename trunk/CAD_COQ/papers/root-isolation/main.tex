\documentclass{article}
\begin{document}
\title{Root isolation for one-variable polynomials}
\author{Yves Bertot, Assia Mahboubi, Fr\'ed\'erique Guilhot}
\section{introduction}

We want to describe an algorithm that isolates the roots of any one
variable polynomial with rational coefficients.
This algorithms constructs a finite list of intervals such that
each interval contains exactly one root of the polynomial and each
root is in one of the intervals.  Such an algorithm can be used as a
basic bloc for other algorithms, for instance to define algebraic numbers or
as a component of cylindrical algebraic decomposition, an algorithm known
to decide systems of inequations between polynomial formulas.

An operation that we will not cover in this paper is an operation
 to reduce the multiplicity of roots: find a new
polynomial that has the same roots, but where each root has
multiplicity one.  This can easily be done by computing the greatest
common divisor between the polynomial and its derivative.  In the
following, we thus assume that our polynomial only has simple roots.

The approach we study is based on Bernstein coefficients.  These
coefficients give a discrete approximation of the behavior of a
polynomial in a given closed interval.  We rely on a sufficient
condition concerning these coefficients (let's call this condition
C1): if the Bernstein coefficients, taken in order, have only one
alternation, then the polynomial is guaranteed to have exactly one
root in the corresponding interval.

A first part of our work is to provide a mechanical proof of condition C1.

We study the relations between the coefficients and roots of a given
polynomial on a given bounded interval and the coefficients and roots
of the image of this polynomial after some transformations. These
transformations relate the coefficients of a polynomial in the
standard monomial basis to the coefficients of another polynomial in
Bernstein bases. They also relate the positive roots of a polynomial
to roots of the other polynomial in some bounded interval.  In fact,
this establishes a relation between condition C1 and Descartes'law of
sign.

Three operations on polynomials establish correspondances between roots of
various polynomials on various intervals.  These operations actually
make it possible to relate Bernstein Coefficients with
plain coefficients of another polynomial, so that the roots of a
polynomial inside a bounded interval are related to the roots of another
polynomial between 0 and positive infinity.  This establishes a
relation between condition C1 and Descartes' law of sign.

In our development, we proved a simplified variant of Descartes's law
of sign for the case where the signs of coefficients of a polynomial
have only one alternation: In this case, the polynomial is guaranteed
to have a single simple root between 0 (excluded) and positive
infinity.

Another part of our work is to describe dichotomy.  Knowing Bernstein
coefficients for a polynomial and a given interval, it is easy to
compute the Bernstein coefficients for the two half intervals, using
the algorithm known as "de Casteljau".  This increases the precision
at which the polynomials behavior is described, so that condition C1
is guaranteed to eventually hold in the dichotomy process.

Most of our proofs were made using only rational numbers as numeric
values.  Thus, we work with type of numbers where equality and
comparison are decidable and the process we described can effectively
be used in a decision procedure.

When considering only rational numbers, the existence of roots takes a
different meaning: if a polynomial has a single simple real root in an
interval, this root may not be rational.  However, we can use a
corresponding property on rational numbers: there exists a
sub-interval inside which the absolute value of the slope is bounded
below, and such that the values of the polynomial at the sub-interval
bounds have opposite signs.  In a similar vein, the intermediate value
theorem does not hold with rational numbers, but a corresponding
statement, expressed as an epsilon-delta property, does.  Our proof
development relies on these facts.

This result can then be used for several purposes.  First it opens the
door for a representation of algebraic numbers as equivalence classes
between pairs consisting of a polynomial with rational coefficients
and an interval.  Root isolation makes it possible to decide when two
such pairs describe the same algebraic number.  Second, it can be used
as a basic bloc for a decision procedure deciding logical formulas
with universal and existential quantification where atomic formulas
are comparisons between polynomial formulas.

Berstein polynomials and de Casteljau's algorithm are intensively used
in computer aided design.

\section{Describing simple roots in the rational setting}
Plan pour cette section
\begin{enumerate}
\item Roots of rational polynomials not necessarily polynomial
\item replace root as a number by a condition on evolution: decompose interval into three part: first part where the polynomial is proved to be strictly negative, second part where the polynomial goes from negative to positive with a slope that is strictly positive, third part where the polynomial is stricly positive
\item In case of Descartes law, impose the slope to be bounded below for all intervals but the first one
\item Using a new form of middle-value theorem: expressed only with rational numbers, applicable only for polynomials
\end{enumerate}
\section{A simple of form of Descartes' law of sign}
\subsection{Mathematical proof}
A lemma on slopes of products of functions: If two positive functions \(f\)
and \(g\) have slopes larger than \(k_f > 0\) and \(k_g > 0\) on the interval
\([a,b]\), then the slope
of the product \(fg\) is larger than \(k_fg(a) + f(a)k_g\).

We first define an invariant that is used for polynomials with only
positive coefficients.  The important characteristics are the
following one (grouped in our development under the name {\tt inv2}):
there exists a positive \(x\) such that \(x * P(x)\)
can be made arbirarily close to 0, the value of \(P(x)\) is positive,
the value of \(P\) in \([0,x]\) is below \(P(x)\), and the polynomial
is increasing above \(x\).  We then show that if any polynomial \(P\)
satisfies these important characteristics, and \(a\) is a non negative
number, then the polynomial \(a + X * P\) also satisfies them.

This is the step case in a proof by induction concerning polynomials
with all non-negative coefficients and at least one positive coefficient.

We can then address polynomials with exactly one alternation.  We want to
show that these polynomials have exactly one root.  We exhibit the three
intervals described in the previous section by giving two values \(x_1\) and \(x_2\) and \(k\) such that the polynomial is negative for every positive \(y\) smaller than \(x_1\), the polynomial is positive in \(x_2\), and the slope between
any two values above \(x_1\) is positive.
\begin{enumerate}
\item Consider a polynomial \(P = a_0 + a_1 x + a_n x^n\)
of degree \(n\), with two natural numbers \(i\), \(j\) such that
\(i < j\), \(a_i < 0\), \(0 < a_j\), and
\[\forall l, l < j \Rightarrow a_k \leq 0\]
\[\forall l, j \leq l \Rightarrow 0 \leq a_l\]
\item Prove that there is exactly one root between 0 and \(+\infty\)
\item Expressed in the following form: there exists \(x > 0\), \(k > 0\),
\(\forall y, 0 < y \leq x \Rightarrow P(y) < 0\) and
\(\forall y z, x \leq y < z \Rightarrow k (z - x) < P(z)-P(x) \).
\item The proof is done by two inductions:
\item First prove that every polynomial with only strictly positive coefficients
is strictly positive and increasing in \((0,+\infty)\)
\item Then prove by induction on \(j\)
\item Consider the truncated polynomial
\(P_t = a_1 + a_2 x + a_j x^{j-1} + a_n x^{n-1}\): we know that \(a_0 \leq 0\)
and there are two cases:
\begin{enumerate}
\item If \(P_t\) has no alternance, then \(a_0 < 0\) and the previous lemma
is satified for \(a_1 x + a_n x ^ n\) , this makes it possible to find
an \(x_1\) such that \(x_1 \times P(x_1) < - a_0\).  Thanks to the lemma
on slopes, the slope of the polynomial \(X \times P\) above \(x_1\) is larger
than \(P_t (x)\), this is enough to find a value \(x_2\) such that 
\(P(x_2)\) is positive.
\item If \(P_t\) has one alternance, then the induction hypothesis provides
a relevant \(x_{1t}\) and \(k_t\) for this polynomial.  We can consider the
value \(y = x_{1t} - a_0/{k_tx_{1t}}\).  This value is larger than \(x_{1t}\).
Thanks to the properties of \(k_t\) we know
that\(y * P_t(y) > k_t (y - x_{1t}) > - a0\) and thus, \(P(y) > 0\)
\item Using the middle-value theorem, we exhibit two new values \(z\) and \(x\),
 such that \(x_{1t} < z < x < y\), \(- k_t x_{1t} < P_t(z) < 0 < P_t(x) < a_0/y\)
\item Obviously, \(P\) is negative below \(x_{1t}\),
\item Thanks to the lemma at the beginning of this section, 
\(X * P_t\) has a slope larger than \(k * x_{1t} + P_t(z) > 0\) between
\(z\) and \(x\), so that \(P\) is negative in this interval,
\item Because \(P_t\) is increasing between \(x_{1t}\) and \(z\), \(P_t\) is
negative in
\item Because \(P_t\) is increasing between \(x_{1t}\) and \(x\), we can show
that \(P\) is negative between \(x_{1t}\) and \(x\)
\item  Thanks to the lemma at the beginning of this section,
the slope of \(P\) above \(x\) is larger than
\(k = P_t(x) + k_t x\).
\end{enumerate}
\end{enumerate}
\subsection{Formal description}
\begin{enumerate}
\item The functions that describe
\end{enumerate}
\end{document}
%%% Local Variables: 
%%% mode: latex
%%% TeX-master: t
%%% End: 
